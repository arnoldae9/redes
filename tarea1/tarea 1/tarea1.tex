\documentclass[a4paper, 12pt]{article}

\usepackage[utf8]{inputenc}
\usepackage{amsmath, amsfonts, amsthm, graphicx, lipsum}
\usepackage[margin = 1in]{geometry}
\usepackage{hyperref}
\hypersetup{
    colorlinks=true,
    linkcolor=blue,
    urlcolor=red,
    pdftitle={Tarea 1},
    }
\usepackage{fancyvrb}
\usepackage{fancyhdr, lastpage}
\pagestyle{fancy}
\lhead{Optimización de flujo en redes}
\rhead{\today}
\cfoot{Page \thepage\ of \pageref{LastPage}}

\usepackage{etoolbox} %Use carefully!
\patchcmd{\chapter}{\thispagestyle{plain}}{\thispagestyle{fancy}}{}{}

\usepackage[Glenn]{fncychap}
%Options: Sonny, Lenny, Glenn, Conny, Rejne, Bjarne, Bjornstrup

\usepackage{xcolor}
\usepackage{tikz}
\usepackage[most]{tcolorbox}

\newtcbtheorem{theo}%
  {Theorem}{}{theorem}
  
\usepackage{siunitx}
\usepackage{setspace}
\onehalfspacing

\usepackage{tikz}
%\usetikzlibrary{arrows,snakes,backgrounds}
\usetikzlibrary{arrows}

\setlength\headheight{15pt}

\author{Lic. Arnoldo Del Toro Peña}
\title{Tarea 1 Optimización de flujo en redes}
\usepackage[spanish]{babel}

\newcommand{\ds}{\displaystyle}

\begin{document}
\maketitle
\begin{enumerate}
      \item {\bfseries Variant of the transportation problem: }
            \\ Considere una variante del problema de transporte, donde la suma de la demanda excede a la suma del suministro, y una penalización $p_j$ por cada unidad no satisfecha en las demandas establecidas.
            \\ Sea $Z:$ el costo total de distribución, y $x_{ij}$  (con $i= 1,2,3 \dots m$, $j=1,2,3 \dots n$, suponiendo que tenemos $m$ nodos de suministro y $n$ nodos de demanda), el número de unidades que se distribuyen del suministro $i$ a la demanda $j$.
            \\ Definamos $c_{ij}$ como el costo por unidad enviada del nodo suministro  $i$ al nodo demanda $j$.
            \\ Definamos $s_{i}$ como el nodo de suministro i.
            \\ Definamos $d_j$ como el nodo de demanda j.
            \\ Función objetivo:
            \[\text{Minimizar} \, Z = \sum_{i=1} ^{m} { \sum_{j=1}^{n} { c_{ij} x_{ij} }  } \]
            Sujeta a:
            \[ \sum_{i=1}^{m} {x_{ij}} = s_{i} \, \text{para toda } i = 1,2,3 \dots m, \]
            \[ \sum_{j=1}^{n} {x_{ij}} = d_{j} \, \text{para toda } j = 1,2,3 \dots n, \]
            y
            \[ x_{ij} \geq 0, \, \text{para toda } i \text{ y } j \]
            El modelo anterior es para un problema en el cual la suma de suministros es igual a la suma de demandas, $\left( \displaystyle \sum_{i=1}^{m} {s_i} = \sum_{j=1}^{n} {d_j} \right)$, para nuestro problema esto no sucede, por lo cual tendremos que agregar un nodo  suministro ''artificial".
            \\ Esto nos provocará tener $m+1$ nodos de suministros, además que tendremos unos costos de penalización por cada unidad que sea transportada del nodo de suministro ''artificial'' hacia uno de los nodos de demanda ya existentes.
            \\ Definamos los costos de penalización como $c_{(m+1),j} = p_j$, y \\ $s_{m+1} = \left( \displaystyle \sum_{j=1}^{n} {d_j} - \sum_{i=1}^{m} {s_i}  \right) > 0.$
            \\ Con lo anterior podemos redefinir el modelo de la siguiente manera:
            \\ Función objetivo:
            \[\text{Minimizar} \, Z = \sum_{i=1} ^{m+1} { \sum_{j=1}^{n} { c_{ij} x_{ij} }  } \]
            Sujeta a:
            \[ \sum_{i=1}^{m+1} {x_{ij}} = s_{i} \, \text{para toda } i = 1,2,3 \dots m+1, \]
            \[ \sum_{j=1}^{n} {x_{ij}} = d_{j} \, \text{para toda } j = 1,2,3 \dots n, \]
            y
            \[ x_{ij} \geq 0, \, \text{para toda } i \text{ y } j \]
            %intento de red:

            \begin{center}
                  \begin{tikzpicture}
                        %suministros
                        \node[draw](s1)[circle] at (0,0) {$s_1$};
                        \node[draw](s2)[circle] at (0,-1) {$s_2$};
                        \node(dots) at (0,-2){\vdots};
                        \node[draw](sm)[circle] at (0,-3) {$s_m$};
                        % nodo extra de suministro
                        \node[draw] (sm+1)[circle] at (0,-5) {$s_{m+1}$};

                        %costos
                        \node(c11) at (2.5,0.25) {$c_{11}$};
                        \node(cmn) at (2.5,-2.75) {$c_{mn}$};

                        %demandas
                        \node[draw](d1)[circle] at (5,0) {$d_1$};
                        \node[draw](d2)[circle] at (5,-1) {$d_2$};
                        \node(dots) at (5,-2){\vdots};
                        \node[draw](dn)[circle] at (5,-3) {$d_n$};

                        %lineas
                        \draw[black, -latex', line width=1pt] (s1) -- (d1);
                        \draw[black, -latex', line width=1pt] (s1) -- (d2);
                        \draw[black, -latex', line width=1pt] (s1) -- (dn);
                        \draw[black, -latex', line width=1pt] (s2) -- (d1);
                        \draw[black, -latex', line width=1pt] (s2) -- (d2);
                        \draw[black, -latex', line width=1pt] (s2) -- (dn);
                        \draw[black, -latex', line width=1pt] (sm) -- (d1);
                        \draw[black, -latex', line width=1pt] (sm) -- (d2);
                        \draw[black, -latex', line width=1pt] (sm) -- (dn);
                        %lineas del nodo extra
                        \draw[red, -latex', line width=1pt] (sm+1) -- (d1);
                        \draw[red, -latex', line width=1pt] (sm+1) -- (d2);
                        \draw[red, -latex', line width=1pt] (sm+1) -- (dn);

                  \end{tikzpicture}
            \end{center}


      \item {\bfseries Dating problem: }
            \\ Definamos un conjunto  $ N = \ds \left\{ h_1, h_2, \dots h_p,m_1,m_2 \dots m_p  \right\}$ como el conjunto de nodos, y un conjunto $A$ como el total de los arcos $(i,j)$ que corresponden a las parejas $(h_i,m_j)$ mutuamente compatibles.
            \\ Buscamos maximizar el número de parejas mutuamente compatibles, tomando en cuenta que si un $h_i$ es asignado a un $m_j$ este ya no puede ser asignado a otro $m_j$ distinto y visceversa, esta relación se puede representar por una red sin dirección.
            \\ El objetivo es encontrar el máximo de parejas mutuamente compatibles.
            \\ Para nuestro problema definiremos $\delta(i)$ como el conjunto de todos los arcos adyacentes en el nodo $i$.
            \\ Además definiremos $x_{ij} = \left\{ \begin{array}{l} \text{1 si seleccionamos el arco } (i,j) \\ \text{0 en otro caso} \end{array} \right. $
            \\ Función objetivo:
            \[ \ds  \text{Maximizar: } \sum_{(i,j) \in A}  {x_{ij}} \]
            \\ Sujeto a:
            \[ \sum_{(i,j) \in \delta(i)} {x_{ij}} \leq 1, \hspace{0.3in} i \in N\]
            \[ x_{ij} \, \in \{ 0,1 \}, \, (i,j) \in A \]
            % segundo intento de red

            \begin{center}
                  \begin{tikzpicture}
                        %hombres
                        \node[draw] (h1) [circle] at (0,0) {$h_1$};
                        \node[draw] (h2) [circle] at (2,0) {$h_2$};
                        \node       (puntos) [circle] at (4,0) {\dots};
                        \node[draw] (hp) [circle] at (6,0) {$h_p$};
                        %mujeres
                        \node[draw] (m1) [circle] at (0,-4) {$m_1$};
                        \node[draw] (m2) [circle] at (2,-4) {$m_2$};
                        \node (puntos)  at (4,-4) {\dots};
                        \node[draw] (mp) [circle] at (6,-4) {$m_p$};

                        %lines
                        \path (h1) edge (m1);
                        \path (h1) edge (m2);
                        \path (h1) edge (mp);

                        \path (h2) edge (m1);
                        \path (h2) edge (m2);
                        \path (h2) edge (mp);

                        \path (hp) edge (m1);
                        \path (hp) edge (m2);
                        \path (hp) edge (mp);

                  \end{tikzpicture}
            \end{center}

            El dibujo anterior solo ilustra las posibles parejas mutuamente compatibles, algunas de las conexiones puede que no existan.




      \item {\bfseries Seat sharing problem: }
            \\ Tenemos $k$ familias, cada familia $k$ tiene $n_k$ integrantes, además contamos con $m$ carros y cada carro tiene una capacidad $c_m$.
            \\ Sea $I = \{1,2,3 \dots k\}$, $J = \{1,2,3 \dots m\}$.
            \\ Sea $u_{ij} = 1$ el flujo máximo de integrantes por familia $i$ al carro $j$, con $i \in I$ y $j \in J$.
            \\ Sea $x_{ij}$ la cantidad de integrantes enviados de la familia $i$ al carro $j$.
            \\ Función objetivo:
            \[ \text{Minimizar }   \sum_{j}  {c_{j}} - \sum_{i} \sum_j {x_{ij}}\]
            Sujeto a:
            \[ x_{ij} \leq u_{ij}, \hspace{0.1in} \forall i \in I, \forall j \in J \]
            \[ \sum_{j \in J} {x_{ij}} = n_i, \hspace{0.1 in} \text{ para cada } i \in I \]
            \[ \sum_{i \in I} {x_{ij}} \leq c_j, \hspace{0.1 in} \text{ para cada } j \in J \]
            \[x_{ij} \geq 0 \, \forall i \in I,\forall j \in J\]
            %tercer intento de red
            \begin{center}
                  \begin{tikzpicture}
                        %familias
                        \node[draw](f1)[circle] at (0,0) {$f_1$};
                        \node[draw](f2)[circle] at (0,-1) {$f_2$};
                        \node(dots) at (0,-2){\vdots};
                        \node[draw](fk)[circle] at (0,-3) {$f_k$};

                        %restricciones uij
                        \node(u11) at (2.5,0.25) {$u_{11}$};
                        \node(ukm) at (2.5,-2.75) {$u_{km}$};

                        %carros
                        \node[draw](c1)[circle] at (5,0) {$c_1$};
                        \node[draw](c2)[circle] at (5,-1) {$c_2$};
                        \node(dots) at (5,-2){\vdots};
                        \node[draw](cm)[circle] at (5,-3) {$c_m$};

                        %lineas
                        \draw[black, -latex', line width=1pt] (f1) -- (c1);
                        \draw[black, -latex', line width=1pt] (f1) -- (c2);
                        \draw[black, -latex', line width=1pt] (f1) -- (cm);
                        \draw[black, -latex', line width=1pt] (f2) -- (c1);
                        \draw[black, -latex', line width=1pt] (f2) -- (c2);
                        \draw[black, -latex', line width=1pt] (f2) -- (cm);
                        \draw[black, -latex', line width=1pt] (fk) -- (c1);
                        \draw[black, -latex', line width=1pt] (fk) -- (c2);
                        \draw[black, -latex', line width=1pt] (fk) -- (cm);

                        %integrantes de las familias
                        \node[draw](n1) at (-1,0) {$n_1$};
                        \node[draw](n2) at (-1,-1) {$n_2$};
                        \node[draw](nk) at (-1,-3) {$n_k$};

                        %capacidades de los carros
                        \node[draw](c1) at (6,0) {$c_1$};
                        \node[draw](c2) at (6,-1) {$c_2$};
                        \node[draw](cm) at (6,-3) {$c_m$};


                  \end{tikzpicture}
            \end{center}



      \item {\bfseries Police patrol problem }
            \\ Definamos el conjunto de nodos tal que $N = \{ p_1,p_2,p_3,s_1,s_2,s_3 \}$ y el conjunto de índices $I= \{1,2,3\}$, $J= \{1,2,3\}$ dónde cada $p_i$ con $i \in I$ denota un precinto, y cada $s_j$ con $j \in J$ denota cada turno.
            \\ Sea $x_{ij}$ el número de patrullas asignadas en el turno $i$ al precinto $j$.
            \\ Sea $u_{ij} = \begin{pmatrix} &s_1&s_2&s_3\\ p_1&3&7&5 \\ p_2&5&7&10 \\ p_3&8&12&10 \end{pmatrix}$ el número máximo de patrullas asignadas en el turno $i$ al precinto $j$.
            \\ Sea $l_{ij} = \begin{pmatrix} &s_1&s_2&s_3\\p_1&2&4&3 \\ p_2&3&6&5 \\ p_3&5&7&6 \end{pmatrix}$ el número mínimo de patrullas asignadas en el turno $i$ al precinto $j$.
            \\ Sea $c_j = \begin{pmatrix} 10 & 14 & 13 \end{pmatrix}$ el número de patrullas necesarias por precinto $j$.
            \\ Sea $d_i = \begin{pmatrix} 10 & 20 & 18 \end{pmatrix}$ el número de patrullas necesarias por turno $i$.
            \\ Función objetivo:
            \[\text{Minimizar } \sum_i \sum_j x_{ij} \]
            Sujeto a:
            \[ \sum_{i}  {x_{ij}} \geq c_j \, \text{ para cada } j \in J\]
            \[ \sum_{j} {x_{ij}} \geq d_i \, \text{ para cada } i \in I\]
            \[ l_{ij}\leq x_{ij} \leq u_{ij} \hspace{0.3in} \text{para cada } i \text{ y } j\]
            \[ x_{ij} \geq 0, \hspace{0.1in} \forall i \in I, \forall j \in J \]


            %cuarto intento de red
            \begin{center}
                  \begin{tikzpicture}
                        %turnos
                        \node[draw](s1)[circle] at (0,0.5)  {$s_1$};
                        \node[draw](s2)[circle] at (0,-1) {$s_2$};
                        \node[draw](s3)[circle] at (0,-2.5) {$s_3$};

                        %restricciones uij
                        \node(u11) at (2.5,0.25)  {$2 \leq x_{11} \leq 3$};
                        \node(u13) at (2.5,-2.75) {$6 \leq x_{13} \leq 10$};

                        %patrullas
                        \node[draw](p1)[circle] at (5,0.5) {$p_1$};
                        \node[draw](p2)[circle] at (5,-1) {$p_2$};
                        \node[draw](p3)[circle] at (5,-2.5) {$p_3$};

                        %lineas
                        \draw[black, -latex', line width=1pt] (s1) -- (p1);
                        \draw[black, -latex', line width=1pt] (s1) -- (p2);
                        \draw[black, -latex', line width=1pt] (s1) -- (p3);
                        \draw[black, -latex', line width=1pt] (s2) -- (p1);
                        \draw[black, -latex', line width=1pt] (s2) -- (p2);
                        \draw[black, -latex', line width=1pt] (s2) -- (p3);
                        \draw[black, -latex', line width=1pt] (s3) -- (p1);
                        \draw[black, -latex', line width=1pt] (s3) -- (p2);
                        \draw[black, -latex', line width=1pt] (s3) -- (p3);

                        %patrullas por turno
                        \node[draw](d1) at (-1,0.5)  {10};
                        \node[draw](d2) at (-1,-1) {20};
                        \node[draw](d3) at (-1,-2.5) {18};

                        %patrullas por precinto
                        \node[draw](c1) at (6,0.5)  {10};
                        \node[draw](c2) at (6,-1) {14};
                        \node[draw](c3) at (6,-2.5) {13};


                  \end{tikzpicture}
            \end{center}
            En la red anterior, todos los $c_j$ y $d_i$ son los valores mínimos que pide el problema.

\end{enumerate}















\end{document}