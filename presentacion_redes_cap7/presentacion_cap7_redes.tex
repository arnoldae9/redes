\documentclass{beamer}
\usepackage[spanish]{babel}
\usepackage{tikz} 
\usepackage{tikz-network}
\usepackage[utf8]{inputenc}
\usetheme{CambridgeUS}
\title{Shortest Aumenting Problem Algorithm}
\author{Lic. Eder Ismael Alanís Fernández\\ Lic. Arnoldo Del Toro Peña }
\institute{Universidad Autónoma de Nuevo León}
\begin{document}

\begin{frame}
 \titlepage
\end{frame}

\begin{frame}
 \frametitle{Correctness of the Algorithm}
\begin{block}<1->{Lema.}
El algoritmo de la ruta de aumento más corta mantiene etiquetas de distancia válidas en cada paso. además, cada operación de reetiquetado (o retirada) aumenta estrictamente la etiqueta de distancia de un nodo. 
\end{block}

\begin{block}<2->{Teorema.}
El algoritmo de ruta de aumento más corto calcula correctamente un flujo máximo.
\end{block}

\end{frame}

\begin{frame}
 \frametitle{Ejemplo 1}
 \tikzstyle{mi nodo}= [circle,draw,blue,fill=blue!15]
 \centering
\begin{tikzpicture}
\Vertex[IdAsLabel]{A} 
\Vertex[IdAsLabel, x=2, y = 2]{B}
\Vertex[IdAsLabel,x=2, y = -2]{C}
\Vertex[IdAsLabel,x=5, y=2]{D}
\Vertex[IdAsLabel,x=5, y=-2]{E}
\Vertex[IdAsLabel,x=7, y=0]{F}
\Edge[Direct,label = 3](A)(B)
\Edge[Direct,label = 3](B)(D)
\Edge[Direct,label = 3](D)(F)
\Edge[Direct,label = 4](A)(C)
\Edge[Direct,label = 2](C)(E)
\Edge[Direct,label = 5](E)(F)
\Edge[Direct,label = 1,distance = 0.3](B)(E)
\Edge[Direct,label = 1,distance = 0.3](C)(D)
\end{tikzpicture}
 
\end{frame}

\begin{frame}
  \frametitle{Ejemplo 1}
 \tikzstyle{mi nodo}= [circle,draw,blue,fill=blue!15]
 \centering
\begin{tikzpicture}
\Vertex[IdAsLabel]{A} 
\Vertex[IdAsLabel, x=2, y = 2]{B}
\Vertex[IdAsLabel,x=2, y = -2]{C}
\Vertex[IdAsLabel,x=5, y=2]{D}
\Vertex[IdAsLabel,x=5, y=-2]{E}
\Vertex[IdAsLabel,x=7, y=0]{F}
\Edge[Direct,label = 3,color = red](A)(B)
\Edge[Direct,label = 3,color = red](B)(D)
\Edge[Direct,label = 3,color = red](D)(F)
\Edge[Direct,label = 4](A)(C)
\Edge[Direct,label = 2](C)(E)
\Edge[Direct,label = 5](E)(F)
\Edge[Direct,label = 1,distance = 0.3](B)(E)
\Edge[Direct,label = 1,distance = 0.3](C)(D)
\end{tikzpicture} \newline
El mínimo es 3.
\end{frame}

\begin{frame}
  \frametitle{Ejemplo 1}
 \tikzstyle{mi nodo}= [circle,draw,blue,fill=blue!15]
 \centering
\begin{tikzpicture}
\Vertex[IdAsLabel]{A} 
\Vertex[IdAsLabel, x=2, y = 2]{B}
\Vertex[IdAsLabel,x=2, y = -2]{C}
\Vertex[IdAsLabel,x=5, y=2]{D}
\Vertex[IdAsLabel,x=5, y=-2]{E}
\Vertex[IdAsLabel,x=7, y=0]{F}
\Edge[Direct,label = 3](B)(A)
\Edge[Direct,label = 3](D)(B)
\Edge[Direct,label = 3](F)(D)
\Edge[Direct,label = 4](A)(C)
\Edge[Direct,label = 2](C)(E)
\Edge[Direct,label = 5](E)(F)
\Edge[Direct,label = 1,distance = 0.3](B)(E)
\Edge[Direct,label = 1,distance = 0.3](C)(D)
\end{tikzpicture} 
\end{frame}

\begin{frame}
  \frametitle{Ejemplo 1}
 \tikzstyle{mi nodo}= [circle,draw,blue,fill=blue!15]
 \centering
\begin{tikzpicture}
\Vertex[IdAsLabel]{A} 
\Vertex[IdAsLabel, x=2, y = 2]{B}
\Vertex[IdAsLabel,x=2, y = -2]{C}
\Vertex[IdAsLabel,x=5, y=2]{D}
\Vertex[IdAsLabel,x=5, y=-2]{E}
\Vertex[IdAsLabel,x=7, y=0]{F}
\Edge[Direct,label = 3](B)(A)
\Edge[Direct,label = 3](D)(B)
\Edge[Direct,label = 3](F)(D)
\Edge[Direct,label = 4,color = red](A)(C)
\Edge[Direct,label = 2,color = red](C)(E)
\Edge[Direct,label = 5,color = red](E)(F)
\Edge[Direct,label = 1,distance = 0.3](B)(E)
\Edge[Direct,label = 1,distance = 0.3](C)(D)
\end{tikzpicture} \newline
El mínimo es 2.
\end{frame}

\begin{frame}
  \frametitle{Ejemplo 1}
 \tikzstyle{mi nodo}= [circle,draw,blue,fill=blue!15]
 \centering
\begin{tikzpicture}
\Vertex[IdAsLabel]{A} 
\Vertex[IdAsLabel, x=2, y = 2]{B}
\Vertex[IdAsLabel,x=2, y = -2]{C}
\Vertex[IdAsLabel,x=5, y=2]{D}
\Vertex[IdAsLabel,x=5, y=-2]{E}
\Vertex[IdAsLabel,x=7, y=0]{F}
\Edge[Direct,label = 3](B)(A)
\Edge[Direct,label = 3](D)(B)
\Edge[Direct,label = 3](F)(D)
\Edge[Direct,label = 2](C)(A)
\Edge[Direct,label = 2,bend = 30](A)(C)
\Edge[Direct,label = 2](E)(C)
\Edge[Direct,label = 3,bend = 30](E)(F)
\Edge[Direct,label = 2](F)(E)
\Edge[Direct,label = 1,distance = 0.3](B)(E)
\Edge[Direct,label = 1,distance = 0.3](C)(D)
\end{tikzpicture} 
\end{frame}

\begin{frame}
  \frametitle{Ejemplo 1}
 \tikzstyle{mi nodo}= [circle,draw,blue,fill=blue!15]
 \centering
\begin{tikzpicture}
\Vertex[IdAsLabel]{A} 
\Vertex[IdAsLabel, x=2, y = 2]{B}
\Vertex[IdAsLabel,x=2, y = -2]{C}
\Vertex[IdAsLabel,x=5, y=2]{D}
\Vertex[IdAsLabel,x=5, y=-2]{E}
\Vertex[IdAsLabel,x=7, y=0]{F}
\Edge[Direct,label = 3](B)(A)
\Edge[Direct,label = 3,color=red](D)(B)
\Edge[Direct,label = 3](F)(D)
\Edge[Direct,label = 2](C)(A)
\Edge[Direct,label = 2,bend = 30,color=red](A)(C)
\Edge[Direct,label = 2](E)(C)
\Edge[Direct,label = 3,bend = 30,color=red](E)(F)
\Edge[Direct,label = 2](F)(E)
\Edge[Direct,label = 1,distance = 0.3,color=red](B)(E)
\Edge[Direct,label = 1,distance = 0.3,color=red](C)(D)
\end{tikzpicture} \newline
El mínimo es 1.
\end{frame}

\begin{frame}
  \frametitle{Ejemplo 1}
 \tikzstyle{mi nodo}= [circle,draw,blue,fill=blue!15]
 \centering
\begin{tikzpicture}
\Vertex[IdAsLabel]{A} 
\Vertex[IdAsLabel, x=2, y = 2]{B}
\Vertex[IdAsLabel,x=2, y = -2]{C}
\Vertex[IdAsLabel,x=5, y=2]{D}
\Vertex[IdAsLabel,x=5, y=-2]{E}
\Vertex[IdAsLabel,x=7, y=0]{F}
\Edge[Direct,label = 3](B)(A)
\Edge[Direct,label = 2](D)(B)
\Edge[Direct,label = 1,bend = 30](B)(D)
\Edge[Direct,label = 3](F)(D)
\Edge[Direct,label = 3](C)(A)
\Edge[Direct,label = 1,bend = 30](A)(C)
\Edge[Direct,label = 2](E)(C)
\Edge[Direct,label = 2,bend = 30](E)(F)
\Edge[Direct,label = 3](F)(E)
\Edge[Direct,label = 1,distance = 0.3](E)(B)
\Edge[Direct,label = 1,distance = 0.3](D)(C)
\end{tikzpicture} 
\end{frame}

\begin{frame}
  \frametitle{Ejemplo 1}
 \tikzstyle{mi nodo}= [circle,draw,blue,fill=blue!15]
 \centering
\begin{tikzpicture}
\Vertex[IdAsLabel]{A} 
\Vertex[IdAsLabel, x=2, y = 2]{B}
\Vertex[IdAsLabel,x=2, y = -2]{C}
\Vertex[IdAsLabel,x=5, y=2]{D}
\Vertex[IdAsLabel,x=5, y=-2]{E}
\Vertex[IdAsLabel,x=7, y=0]{F}
\Edge[Direct,label = 3](B)(A)
\Edge[Direct,label = 2](D)(B)
\Edge[Direct,label = 1,bend = 30](B)(D)
\Edge[Direct,label = 3](F)(D)
\Edge[Direct,label = 3](C)(A)
\Edge[Direct,label = 1,bend = 30,color = red](A)(C)
\Edge[Direct,label = 2](E)(C)
\Edge[Direct,label = 2,bend = 30](E)(F)
\Edge[Direct,label = 3](F)(E)
\Edge[Direct,label = 1,distance = 0.3](E)(B)
\Edge[Direct,label = 1,distance = 0.3](D)(C)
\end{tikzpicture} \newline
El algoritmo se detiene.
\end{frame}

\end{document}
